\documentclass[12pt,a4paper]{article}
\usepackage[latin1]{inputenc}
\usepackage{amsmath}
\usepackage{amsfonts}
\usepackage{amssymb}
\usepackage{graphicx}
\usepackage{mathtools}
\usepackage{url}
\begin{document}
\title {Conditional Probability\\ Problem/Solution Based Approach}
\author {Putu Wiramaswara Widya}
\maketitle

\section{The Discrete Case}
\subsection{Exercise 1}

If $X$ and $Y$ are both discrete, show that $\sum_xp_{x|y}(x|y) = 1$ for all y such that $p_y(y)>0$.

\subsubsection{Solution}
We have know that conditional probability of $P(A|B) = \frac{P(A\intersectB)}{P(B)}$. We find the extension of this formula for discrete random variable as :
\[
    \sum_xp_{x|y}(x|y) = \frac{\sum_xp(x,y)}{p_y(y)} 

\]

The $\sum_xp(x,y)$ equal to $p_y(y)$ due to fact that summation of $p(x,y)$ for every x is equal to marginal of $y$, so that the formula above become :
\[
    \frac{p_y(y)}{p_y(y)} = 1
\]

\subsection{Exercise 2}
Let $X_1$ and $X_2$ be independent geometric random variables having the same parameter $p$. Guess the value of :

\[
    P\{X_1 = i | X_1 + X_2 = n\}

\]

\subsubsection{Hint}
Suppose a coin having probability $p$ of coming up heads is continually flipped. If the second head occur on flip number $n$, what is the conditional probability that the first head was on flip number $i$, $i = 1, \dots, n-1$?

\subsubsection{Solution}

\subsection{Exercise 3}
The join probability mass function of $X$ and $Y$ and $p(x,y)$, is given by :

\[
p(1,1) = 1/9, p(2,1) = 1/3, p(3,1) = 1/9
\]
\[
p(1,2) = 1/9, p(2,2) = 0, p(3,2) = 1/8
\]
\[
p(1,3) = 0, p(2,3) = 1/6, p(3,3) = 1/9
\]

Compute $E[X|Y=i]$ for $i=1,2,3$.

\subsubsection{Solution}

\[
E[X|Y=1] = 1(\frac{1/9}{5/9}) + 2(\frac{3/9}{5/9}) + 3(\frac{1/9}{5/9}) =  2
\]

\[
E[X|Y=2] = 1(\frac{2/18}{2/18}) + 2(\frac{0}{2/18}) + 3(\frac{1/18}{2/18}) = 5/3
\]

\[
E[X|Y=3] = 1(\frac{0}{5/18}) + 2(\frac{3/18}{5/18}) + 3(\frac{2/18}{5/18}) = 12/5
\]


\subsection{Exercise 4}

In Exercise 3, are the random variable $X$ and $Y$ independent?

\subsubsection{Solution}

We can do random check within some value of $X$ and $Y$ wheter or not $P\{X=a,Y=b\} = P\{X=a\} P\{Y=b\}$. Suppose $a=1, b=1$ :

\[
P(X=1) = 2/9, P(Y=1) = 5/9, P(X=1, Y=1) = 2/9 * 5/9 = 10/9 \neq 1/9
\]

\subsection{Exercise 5}

An urn contains three white, six red, and five black balls. Six of these balls are randomly selected from the urn. Let $X$ and $Y$ denote respectively the number of white and black balls selected. Compute the conditional probability mass function of $X$ given that $Y=3$. Also compute $E[X|Y=1]$.

\subsubsection{Solution}

We will find the probability mass function for $P(X=i|Y=3)$ in which $X$ is number of white ball selected from choosing six balls. Given that we have selected three black balls, the binomial choosing will be only choosing 3 between 3 white and 6 red.

\[
P(X=i, Y=3) = 
\frac{\begin{array}{c}3\\i\end{array}\begin{array}{c}6\\3-i\end{array}}{\begin{array}{c}9\\3\end{array}}
\]


\subsection{Exercise 6}

Repeat Exercise 5 but under assumption that when a ball is selected its color is noted, and it is then replaced in the urn before the next selection is made.


\subsubsection{Solution}

$[(x)]$

\section{The Continous Case}
\section{Computing Expectation by Conditioning}
\section{Computing Probabilities by Conditioning}

\end{document}
